\documentclass{article}
%% Packages
%\usepackage{lmodern}
\usepackage{amsthm}
\usepackage{amssymb}
\usepackage{amsmath}
\usepackage{amsfonts}
\usepackage{booktabs}
\usepackage{color}
\usepackage{multirow}
\usepackage{array}
\usepackage{pgfplots}
\usepackage{pgffor}
\pgfplotsset{compat=1.9}
\usepackage{dsfont}
\usepackage{mathrsfs}
%\usepackage{algorithmic,algorithm}
\usepackage{algorithm}
\usepackage{algpseudocode}
\usepackage{comment}
\usepackage{lscape}
\usepackage{ifthen}
\usepackage{subcaption}
\usepackage{longtable}
\usepackage{comment}
\usepackage{listings}
%% New Commands

% Equations
\newcommand{\ali}[1]{\begin{align} #1 \end{align}} %align-environment
\newcommand{\alin}[1]{\begin{align*} #1 \end{align*}} %align-environment without numbering
\newcommand{\alil}[2]{\begin{align}\label{#1} #2 \end{align}} %align-environment with label (for 1-line eqs.)

\newcommand{\problem}[5]{\setlength{\arraycolsep}{0.0em}
\begin{equation}\label{#1}
\begin{array}{ll} 
#5 = \hspace*{0.1cm} & \underset{#2}{\min}  \hspace{0.15cm}#3\\ 
\text{s.t.} & #4 
\end{array}\end{equation}}

\newcommand{\MPCproblem}[5]{\begin{equation}\label{#1}
\begin{array}{ll} 
\underset{#2}{\min}  &\hspace{0.15cm}#3\\ 
\text{s.t.} & #4 
\end{array}\end{equation}}

% Sum and Limit
\newcommand{\smlm}[3]{\sum \limits_{#1}^{#2} #3 } %sum with limits #1 and #2
\newcommand{\itlm}[4]{\int \limits_{#1}^{#2} #3 \df #4} %int #3 with limits #1 and #2 and differential #4
\newcommand{\mini}[2]{\min \limits_{#1} #2 } %Minimum with #1 as limit and #2 as argument
\newcommand{\mxl}[2]{\max \limits_{#1} #2 } %Maximum with #1 as limit and #2 as argument
\newcommand{\abso}[1]{\left| #1 \right|}

% Brackets and Vector
\newcommand{\sbr}[1]{\left[#1\right]} %square brackets
\newcommand{\rbr}[1]{\left(#1\right)} %round brackets
\newcommand{\vect}[1]{\rbr{\begin{array}{c} #1 \end{array}}} %General vector with entry #1
\newcommand{\vectsq}[1]{\begin{bmatrix} #1 \end{bmatrix}}

% Table
\newcommand{\otoprule}{\midrule [\heavyrulewidth]} %From Martina
\newcommand{\tbl}[5]{\begin{table}[!htb]\caption{#1}\label{#2}\centering\begin{tabular}{#3}\toprule #4 \\ \otoprule #5\\ \bottomrule\end{tabular}\end{table}} %General definition of a table: #1:caption, #2:label, #3:column definition, #4:header, #5:main body
\newcommand{\longtbl}[5]{\begin{longtable}[!htb]{#3}\caption{#1}\label{#2}\centering \toprule #4 \\ \otoprule #5\\ \bottomrule\end{longtable}} %General definition of a table: #1:caption, #2:label, #3:column definition, #4:header, #5:main body
\newcommand{\tblt}[5]{\begin{table*}[!htb]\caption{#1}\label{#2}\centering\begin{tabular}{#3}\toprule #4 \\ \otoprule #5\\ \bottomrule\end{tabular}\end{table*}} %General definition of a table: #1:caption, #2:label, #3:column definition, #4:header, #5:main body
\newcommand{\newcr}[3]{$#1$ && $X = \rbr{#2}$ \newline $Y = \sbr{#3}^T$\\}

% Differential
\newcommand{\df}{\text{d}}
\newcommand{\dff}[2]{\frac{\text{d} #1}{\text{d} #2}}
\newcommand{\dffz}[2]{\frac{\text{d}^2 #1}{\text{d} #2 ^2}}
\newcommand{\pdff}[2]{\frac{\partial #1}{\partial #2}} %Partial differential
\newcommand{\pdffz}[2]{\frac{\partial^2 #1}{\partial #2 ^2}} %Second partial differential

% Mathematical Spaces
\newcommand{\spacen}[1]{\operatorname{\mathbb{#1}}} %Mathematical space like R, C, K etc.
\newcommand{\spaceo}[2]{\operatorname{\mathbb{#1}}^{#2}} %Mathematical space like R^n, C^m etc.
\newcommand{\spacet}[3]{\operatorname{\mathbb{#1}}^{#2 \times #3}} %Mathematical space like R^{nxn}, C^{mxn} etc.

% Figures
\newcommand{\fig}[2]{\begin{figure}[!htb]\centering \includegraphics[width = #1 \textwidth]{#2} \end{figure}} %Figure without label or caption
\newcommand{\figc}[4]{\begin{figure}[!htp]\centering \includegraphics[width = #1 \textwidth]{#2} \caption{#3}\label{#4} \end{figure}} %Figure with label and caption

% Curly Array Brackets
\newcommand{\lftcur}[2]{\left\{ \begin{array}{#1} #2 \end{array}\right.} %#1 is the column structure and #2 the entries
\newcommand{\rgtcur}[2]{\left. \begin{array}{#1} #2 \end{array}\right\} } %#1 is the column structure and #2 the entries
\newcommand{\lrcur}[2]{\left\{ \begin{array}{#1} #2 \end{array}\right\} } %#1 is the column structure and #2 the entries

%% New Theorems
\theoremstyle{remark}
\newtheorem{rem}{Remark}
\newtheorem*{pf}{Proof}
\theoremstyle{definition}
\newtheorem{thm}{Theorem}
\newtheorem{defn}{Definition}
\newtheorem{lem}{Lemma}

%% Tikz stuff
\newenvironment{customlegend}[1][]{%
    \begingroup
    \csname pgfplots@init@cleared@structures\endcsname
    \pgfplotsset{#1}%
}{%
    \csname pgfplots@createlegend\endcsname
    \endgroup
}%
\def\addlegendimage{\csname pgfplots@addlegendimage\endcsname}

\newcommand{\addlegendimageintext}[1]{%
    \tikz {
        \begin{customlegend}[
            legend entries={\empty},
            legend style={
                draw=none,
                inner sep=0pt,
                column sep=0pt,
                nodes={inner sep=0pt}}]
        \addlegendimage{#1}
        \end{customlegend}
    }%
}

% Define flow chart styles
\tikzstyle{decision} = [diamond, shape aspect = 1.7, draw, align=center, node distance=3cm]
\tikzstyle{block} = [rectangle, draw, text width=5em, text centered, rounded corners, minimum height=4em]
\tikzstyle{line} = [draw, -latex']
\tikzstyle{cloud} = [draw, ellipse, node distance=3cm, minimum height=2em]

%text width=4.5em, 
\author{Richard Oberdieck, Ruben Menke}
\title{Math of BarMeUp}

\begin{document}
\maketitle

\section{Background}
BarMeUp is about using analytics and math programming to optimally stack your bar so that you get drunk. Kind of. Basically, we have a database of drinks and ingredients needed to make those drinks and then we ask this data various questions using math programming. That's it at least for now. Let's see what else we can do once we're done with this.

First off though, we need to define the two main components:
\begin{description}
\item[Drink]{A drink $d$ is consisting of ingredients $i$ (with a needed quantity $q_i$) prepared in a specific way (let's say algorithm $a$). This will require a set of tools $t$. All of this information is stored and can be accessed to formulate an appropriate optimization problem.}
\item[Ingredient]{An ingredient $i$ has a specific name as well as a cost, availability ("high", "medium" and "low") as well as usage speed (= how quickly it needs to be finished before it goes bad) in days.}
\end{description}

\rem{For now we will not go into the algorithms and tools needed for the drink, but that may be something to consider in the future.}

So now we'll dive into the different questions we would like to have answered.

\section{Basic model}
The most basic question is to minimize the number of ingredients to  be able to make $n$ cocktails. We assume that we have the following given:
\begin{itemize}
\item{A number of ingredients already present}
\item{A maximum number $n$ of cocktails we would like to make}
\end{itemize}

\rem{This is kind of the easiest thing to answer, but we can extend thus arbitrarily.}

This results in the following optimization problem:
\begin{equation}\label{general1}
\begin{array}{rl}
\text{minimize} & \smlm{i}{\quad}{x_i} \\
\text{subject to} & \smlm{i\in I_d}{}{z_i} \geq \abso{I_d}y_d \\
& z_i = x_i + p_i \\
& \smlm{d}{}{y_d} = n \\
& p_i = 1, \hspace{0.15cm} \text{if ingredient is present} \\
& p_i = 0, \hspace{0.15cm} \text{if ingredient is not present} \\
& x_i,y_d,z_i,p_i \in \{0,1\}
\end{array}
\end{equation}
where $I_d$ is the set of ingredients needed to make drink $d$ and$\abso{\cdot}$ denotes the cardinality of a set.
\subsection{Extensions of problem (\ref{general1})}
The extensions that come to mind are the following:
\begin{itemize}
\item{I want to minimize my cost, not the number of ingredients}
\item{I want to weight the ingredients by how quickly they have to be consumed.}
\item{I want to know which long term items I should buy to build my bar up (and ignore the perishable ingredients). An extension of this could be to only consider perishable ingredients that are easily available.}
\end{itemize}


\end{document}
